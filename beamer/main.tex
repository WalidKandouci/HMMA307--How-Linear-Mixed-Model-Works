\documentclass[unknownkeysallowed]{beamer}
\usepackage[french,english]{babel}
\input{Beamer_js}
\input{shortcuts_js}
%\usepackage{./OrganizationFiles/tex/sty/shortcuts_js}
\usepackage{csquotes}
\graphicspath{{./images/}}

\addbibresource{Bibliographie.bib}
\usepackage{enumerate}

\usepackage[T1]{fontenc}
\usepackage[utf8x]{inputenc}
%\usepackage{dsfont}
%\usepackage{bbold}
%\usepackage{stmaryrd}
%\languagepath{French}
%\usepackage{xcolor}
%\usepackage{pgf}
%\usepackage{tikz}



\usepackage[procnames]{listings}
% \usepackage{setspace} % need for \setstretch{1}
\lstset{%
language   = python,%
 % basicstyle = \ttfamily\setstretch{1},%
basicstyle = \ttfamily,%
columns    = flexible,%
keywordstyle=\color{javared},
firstnumber=100,
frame=shadowbox,
showstringspaces=false,
morekeywords={import,from,class,def,for,while,if,is,in,elif,
else,not,and,or,print,break,continue,return,True,False,None,access,
as,del,except,exec,finally,global,import,lambda,pass,print,raise,try,assert,!=},
keywordstyle={\color{javared}\bfseries},
commentstyle=\color{javagreen}, %vsomb_col white comments
morecomment=[s][\color{javagreen}]{"""}{"""},
upquote=true,
%% style for number
numbers=none,
resetmargins=true,
xleftmargin=10pt,
linewidth= \linewidth,
numberstyle=\tiny,
stepnumber=1,
numbersep=8pt, %
frame=shadowbox,
rulesepcolor=\color{black},
procnamekeys={def,class},
procnamestyle=\color{oneblue}\textbf,
literate={á}{{\'a}}1
{à}{{\`a }}1
{ã}{{\~a}}1
{é}{{\'e}}1
{ê}{{\^e}}1
{è}{{\`e}}1
{í}{{\'i}}1
{î}{{\^i}}1
{ó}{{\'o}}1
{õ}{{\~o}}1
{ô}{{\^o}}1
{ú}{{\'u}}1
{ü}{{\"u}}1
{ç}{{\c{c}}}1
}










\begin{document}


\begin{frame}[noframenumbering]
\thispagestyle{empty}
\bigskip
\bigskip
\begin{center}{
\LARGE\color{marron}
\textbf{Les Modèles Linéaires Mixtes}
\textbf{ }\\
\vspace{0.5cm}
}

\color{marron}
\textbf{HMMA 307 : Modèles Linéaires Avancés}
\end{center}

\vspace{0.5cm}

\begin{center}
\textbf{KANDOUCI Walid} \\
\vspace{0.1cm}
\url{https://github.com/WalidKandouci/HMMA307--How-Linear-Mixed-Model-Works.git}\\
\vspace{0.5cm}
Université de Montpellier \\
\end{center}

\centering
\includegraphics[width=0.13\textwidth]{Logo.pdf}
\end{frame}






%%%%%%%%%%%%%%%%%%%%%%%%%%%%%%%%%%%%%%%%%%%%%%%%%%%%%%%%%%%%%%%%%%%%%%%%%%%%%%%
%%%%%%%%%%%%%%%%%%%%%%             Headers               %%%%%%%%%%%%%%%%%%%%%%
%%%%%%%%%%%%%%%%%%%%%%%%%%%%%%%%%%%%%%%%%%%%%%%%%%%%%%%%%%%%%%%%%%%%%%%%%%%%%%%


% plan de la présentation
\begin{frame}
\frametitle{Sommaire}
\tableofcontents
\end{frame}


\section{Introduction}
\begin{frame}
\frametitle{Introduction}
\begin{itemize}
    \item \textbf{Modèles mixtes:} des modèles comportant à la fois des facteurs à effets fixes (ces effets entrant dans la définition de la moyenne du modèle) et des facteurs à effets aléatoires (ces effets entrant, quant à eux, dans la définition de la variance du modèle)
\end{itemize}
    $$y=X \beta+Zu+\epsilon$$
    \begin{itemize}
    \item[$\bullet$] $X$: Matrice $(N\times p)$ 
    \item[$\bullet$] $\beta$: Les coefficients de régression pour chaque variable indépendante du modèle
    \item[$\bullet$] $epsilon$: Contient les erreurs (résidus) du modèle
    \item[$\bullet$] $Z$: Matrice qui contient les valeurs observées pour chaque individu $N$ pour chaque covariable q des effets aléatoires.
    \item[$\bullet$] $u$ est un vecteur $(q\times 1)$ qui contient les effets aléatoires des $q$ covariables de $Z$

\end{itemize}
\end{frame}






\section{Exemple: sleepstudy}

\begin{frame}
\frametitle{Exemple: sleepstudy}
\begin{center}
\begin{tabular}{ c c c c }
\hline
 & Reaction & Days & Subject \\
\hline
1 & 249.5600 & 0 & 308 \\
2 & 258.7047 & 1 & 308 \\
3 & 250.8006 & 2 & 308 \\
4 & 321.4398 & 3 & 308 \\
5 & 356.8519 & 4 & 308 \\
\hline
\end{tabular}
\end{center}
 \begin{itemize}
     \item \textbf{180 individus $\times$ 3 variables}
     \item[$\bullet$] \textbf{Reaction:} Temps de réaction en moyenne (en milisecondes)
    \item[$\bullet$] \textbf{Days:} Nombre de jours de privation de sommeil
    \item[$\bullet$] \textbf{Subject:} Numéro du sujet sur lequel l'observation a été faite.
 \end{itemize}

\end{frame}


\begin{frame}{Nos variables "Reactions" et "Days"}
    \begin{figure}[H]
\centering
\begin{subfigure}{.5\textwidth}
  \centering
  \includegraphics[width=1\linewidth]{Images/figure3.pdf}
  \caption{Distribution des réactions}
\end{subfigure}%
\begin{subfigure}{.5\textwidth}
  \centering
  \includegraphics[width=1\linewidth]{Images/figure4.pdf}
  \caption{Distribution des jours}
\end{subfigure}
\caption{Analyse de nos variables "Reaction" et "Days"}
\end{figure}
   \begin{itemize}
     \item \textbf{Reaction:} Temps de réaction en moyenne (en milisecondes)
    \item \textbf{Days:} Nombre de jours de privation de sommeil
 \end{itemize}
\end{frame}

\begin{frame}{Évolution du temps de réaction en fonctiondes jours}
    \begin{figure}[H]
    \centering
    \includegraphics[width=10cm, height=5cm]{Images/figure.pdf}
    \caption{violins-plots de nos données, jours /réactions.}
\end{figure}
    \begin{itemize}
     \item Le temps réaction par rapport aux jours a une tendance à la hausse avec des variations entre les jours et les individus.
    
 \end{itemize}
\end{frame}

\begin{frame}{Évolution du temps de réaction en fonction des jours}
    \begin{figure}[H]
    \centering
    \includegraphics[width=7.5cm, height=5cm]{Images/figure5.pdf}
    \caption{Représentation du temps de réaction en fonctions des jours}
\end{figure}
\begin{itemize}
     \item Le temps réaction par rapport aux jours a une tendance à la hausse avec des variations entre les jours et les individus.
 \end{itemize}
\end{frame}

\section{Application}
\begin{frame}{Application}
   \begin{itemize}
       \item Nous allons implémenter maintenant les méthodes \textbf{MML}, \textbf{OLS} et \textbf{GLM}
       \item  Nous allons comparer leurs erreur quadratique moyenne \textbf{(RMSE)}
       \item Analayser le résultat obtenue grace a la méthode \textbf{MML}
   \end{itemize}
\end{frame}


\begin{frame}[fragile]{Application}

\begin{lstlisting}[language=Python]
# OLS
modelOLS = smf.ols("Reaction ~ Days", 
                  data, groups=data["Subject"])
resultOLS = modelOLS.fit()
print(resultOLS.summary())
# GLM
modelGLM = smf.glm("Reaction ~ Days", 
                  data, groups=data["Subject"])
resultGLM = modelGLM.fit()
print(resultGLM.summary())
#LMM
md = smf.mixedlm("Reaction ~ Days", 
                data, groups=data["Subject"])
mdf = md.fit()
print(mdf.summary())
\end{lstlisting}

\end{frame}


\begin{frame}{Application}
\begin{center}
\begin{tabular}{ c c c c }
\multicolumn{4}{c}{OLS Regression Results} \\
\end{tabular}
\begin{tabular}{ c c c c c c }
 & Coef. & Std.Err. & z & $P>|z|$  & $[0.025  0.975]$\\
\hline
Intercept & 251.4051 & 6.610 & 38.033 & 0.000 & 238.361  264.449\\
Days & 10.4673 & 1.238 & 8.454 & 0.000 & 8.024  12.911\\
\hline
\end{tabular}
\end{center}
\begin{center}
\begin{tabular}{ c c c c }
\multicolumn{4}{c}{Generalized Linear Model Regression Results} \\
\end{tabular}
\begin{tabular}{ c c c c c c }
 & Coef. & Std.Err. & z & $P>|z|$  & $[0.025  0.975]$\\
\hline
Intercept & 251.4051 & 6.610 & 38.033 & 0.000 & 238.449  264.361\\
Days & 10.4673 & 1.238 & 8.454 & 0.000 & 8.040  12.894\\
\hline
\end{tabular}
\end{center}


\begin{center}
\begin{tabular}{ c c c c }
\multicolumn{4}{c}{Mixed Linear Model Regression Results} \\
\end{tabular}
\begin{tabular}{ c c c c c c }
 & Coef. & Std.Err. & z & $P>|z|$  & $[0.025  0.975]$\\
\hline
Intercept & 251.405 & 9.747 & 25.794 & 0.000 & 232.302 270.508\\
Days & 10.467 & 0.804 & 13.015 & 0.000 & 8.891  12.044\\
Group Var & 1378.176 & 17.156 & & & \\
\hline
\end{tabular}
\end{center}
\end{frame}





\begin{frame}{Comparaison des RMSE}
$$ RMSE=\sqrt{\sum_{i=1}^{n} \frac{\left(\hat{y}_{i}-y_{i}\right)^{2}}{n}} $$

\begin{center}
\begin{tabular}{ c c }
\hline
Method & RMSE\\
\hline
LMM	& 29.410624\\
GLM	& 47.448898\\
OLS	& 47.448898\\
\hline
\end{tabular}
\end{center}

\begin{itemize}
    \item On remarque que l'erreur quadratique moyenne du \textbf{MML} est la plus petite.
\end{itemize}
\end{frame}



\begin{frame}{LMM Résultat}
     \begin{figure}[H]
    \centering
    \includegraphics[width=7.5cm, height=5cm]{Images/figure6.pdf}
    \caption{Représentation du temps de réaction en fonctions des jours}
    \end{figure}
\begin{itemize}
    \item Le temps de réactions étant mieux répartis en fonction des jours.
\end{itemize}
\end{frame}

\section{Conclusion}
\begin{frame}{Conclusion}
\begin{itemize}
    \item Les modèles linéaires mixtes sont utilisé pour tenir compte de la non-indépendance des données
    \item le modèle fourni généralement un meilleur ajustement et expliquent plus de variation dans les données.
\newpage
    
\end{itemize}
\end{frame}

\end{document}

