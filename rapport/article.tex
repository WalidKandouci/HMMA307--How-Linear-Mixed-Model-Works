\documentclass[a4paper,10pt]{article}
\usepackage[margin=2.5cm]{geometry}

\usepackage{amssymb,amsmath,amsthm}
\usepackage{color}
\usepackage{enumitem}
\usepackage{dsfont}
\usepackage{bm}

\usepackage[authoryear]{natbib}
\usepackage{float}
\usepackage[T1]{fontenc}
\usepackage[utf8]{inputenc}
\usepackage[french]{babel} 
\usepackage{amsmath}
\usepackage{amssymb}
%\usepackage[top=1.5cm, bottom=1.5cm, left=1.5cm, right=1.5cm]{geometry}
\usepackage{graphicx}
\usepackage{float}
\usepackage{multicol}
\usepackage{lipsum}
\usepackage{ragged2e}
\usepackage{eurosym}
\usepackage{indentfirst}
\usepackage{minted}
\usepackage{titlesec}
\usepackage{pifont}
\usepackage{url}
\usepackage{epsf}



\newtheorem{theorem}{Theorem}
\newtheorem{proposition}{Proposition}
\newtheorem{cor}{Corollary}
\theoremstyle{definition}
\newtheorem{definition}{Definition}
\newtheorem{remark}{Remark}
\newtheorem{example}{Example}
\newtheorem{claim}{Claim}
\newtheorem{lemma}{Lemma}


%%%%%%%%%%%%%%%%%%%%%%%%%%%%%%%%%%%%%%%%%%%%%%%%%%%%%%%%%%%%%%%%%%%%%%%%%%%%%%%
% Algorithms
%%%%%%%%%%%%%%%%%%%%%%%%%%%%%%%%%%%%%%%%%%%%%%%%%%%%%%%%%%%%%%%%%%%%%%%%%%%%%%%

\usepackage{algorithm}
\usepackage{algorithmic}
\usepackage[titlenumbered,ruled,noend,algo2e]{algorithm2e}
\newcommand\mycommfont[1]{\footnotesize\ttfamily\textcolor{blue}{#1}}
\SetCommentSty{mycommfont}
\SetEndCharOfAlgoLine{}


%%%%%%%%%%%%%%%%%%%%%%%%%%%%%%%%%%%%%%%%%%%%%%%%%%%%%%%%%%%%%%%%%%%%%%%%%%%%%%%
% Code
%%%%%%%%%%%%%%%%%%%%%%%%%%%%%%%%%%%%%%%%%%%%%%%%%%%%%%%%%%%%%%%%%%%%%%%%%%%%%%%


\usepackage{fancyvrb}                  % for fancy verbatim
\usepackage{textcomp}
\usepackage[space=true]{accsupp}
% requires the latest version of package accsupp
\newcommand{\copyablespace}{
    \BeginAccSupp{method=hex,unicode,ActualText=00A0}
\ %
    \EndAccSupp{}
}
\usepackage[procnames]{listings}
% \usepackage{setspace} % need for \setstretch{1}
\lstset{%
language   = python,%
 % basicstyle = \ttfamily\setstretch{1},%
basicstyle = \ttfamily,%
columns    = flexible,%
keywordstyle=\color{javared},
firstnumber=100,
frame=shadowbox,
showstringspaces=false,
morekeywords={import,from,class,def,for,while,if,is,in,elif,
else,not,and,or,print,break,continue,return,True,False,None,access,
as,del,except,exec,finally,global,import,lambda,pass,print,raise,try,assert,!=},
keywordstyle={\color{javared}\bfseries},
commentstyle=\color{javagreen}, %vsomb_col white comments
morecomment=[s][\color{javagreen}]{"""}{"""},
upquote=true,
%% style for number
numbers=none,
resetmargins=true,
xleftmargin=10pt,
linewidth= \linewidth,
numberstyle=\tiny,
stepnumber=1,
numbersep=8pt, %
frame=shadowbox,
rulesepcolor=\color{black},
procnamekeys={def,class},
procnamestyle=\color{oneblue}\textbf,
literate={á}{{\'a}}1
{à}{{\`a }}1
{ã}{{\~a}}1
{é}{{\'e}}1
{ê}{{\^e}}1
{è}{{\`e}}1
{í}{{\'i}}1
{î}{{\^i}}1
{ó}{{\'o}}1
{õ}{{\~o}}1
{ô}{{\^o}}1
{ú}{{\'u}}1
{ü}{{\"u}}1
{ç}{{\c{c}}}1
}

\definecolor{javared}{rgb}{0.6,0,0} % for strings
\definecolor{javagreen}{rgb}{0.25,0.5,0.35} % comments
\definecolor{javapurple}{rgb}{0.5,0,0.35} % keywords
\definecolor{javadocblue}{rgb}{0.25,0.35,0.75} % javadoc
\definecolor{marron}{rgb}{0.64,0.16,0.16}
\definecolor{orange_js}{RGB}{230,159,0}

\usepackage{times} % use Times

%\usepackage{../sty/shortcuts_js} % possibly adapted from https://github.com/josephsalmon/OrganizationFiles/sty/shortcuts_js.sty

%%%%%%%%%%%%%%%%%%%%%%%%%%%%%%%%%%%%%%%%%%%%%%%%%%%%%%%%%%%%%%%%%%%%%%%%%%%%%%%
% IMAGES
%%%%%%%%%%%%%%%%%%%%%%%%%%%%%%%%%%%%%%%%%%%%%%%%%%%%%%%%%%%%%%%%%%%%%%%%%%%%%%%

% Use prebuiltimages/ for images extracted from code (e.g. python)
% or to share images built from a software not available by the whole team (say matlab .fig, or inskcape .svg).
% .svg files should be stored in dir srcimages/ and built from moosetex if needed:
% https://www.charles-deledalle.fr/pages/moosetex.php
% NEVER (GIT) versions files in images/ : only prebuiltimages/ & srcimages/ !

\usepackage{graphicx} % For figures
\graphicspath{{images/}, {prebuiltimages/}}
\usepackage{subcaption}


%%%%%%%%%%%%%%%%%%%%%%%%%%%%%%%%%%%%%%%%%%%%%%%%%%%%%%%%%%%%%%%%%%%%%%%%%%%%%%%
% For citations
%%%%%%%%%%%%%%%%%%%%%%%%%%%%%%%%%%%%%%%%%%%%%%%%%%%%%%%%%%%%%%%%%%%%%%%%%%%%%%%


\usepackage{cleveref} % mandatory for no pbs with hyperlinks theorem etc...
\crefformat{equation}{Eq.~(#2#1#3)} % format for equations
\Crefformat{equation}{Equation~(#2#1#3)} % format for equations


%%%%%%%%%%%%%%%%%%%%%%%%%%%%%%%%%%%%%%%%%%%%%%%%%%%%%%%%%%%%%%%%%%%%%%%%%%%%%%%
% Header and document start
%%%%%%%%%%%%%%%%%%%%%%%%%%%%%%%%%%%%%%%%%%%%%%%%%%%%%%%%%%%%%%%%%%%%%%%%%%%%%%%


%\author{Walid Kandouci}
%\title{Comment fonctionne du modèle mixte linéaire
%}

\begin{document}

\begin{titlepage}
\newcommand{\HRule}{\rule{\linewidth}{0.5mm}}
\center
\textsc{\LARGE
Université de Montpellier
} \\[1cm]
\includegraphics[scale=0.4]{umontpellier_logo} \\[1cm]
\HRule \\[0.4cm]
{ \Huge \bfseries Les Modèles Linéaires Mixtes \\[0.15cm] }
\HRule \\[0.4cm]
 \Large  HMMA 307: Modèles Linéaires Avancés \\[13cm]
 

\LARGE Walid KANDOUCI

\end{titlepage}

%\maketitle

%\vskip 0.3in

%\begin{figure}[h] % h stands for here, ! forces even more...
%	\centering
%	\includegraphics[width=0.2\textwidth]{umontpellier_logo}
%	\caption{Illustration of a prebuiltimage available.}
%	\label{fig:umontpellier_logo}
%\end{figure}

%\begin{abstract}
%\input{content/abstract}
%\end{abstract}

\newpage

\tableofcontents

\newpage




%%%%%%%%%%%%%%%%%%%%%%%%%%%%%%%%%%%%%%%%%%%%%%%%%%%%%%%%%%%%%%%%%%%%%%%%%%%%%%%
% Sections in separated files
%%%%%%%%%%%%%%%%%%%%%%%%%%%%%%%%%%%%%%%%%%%%%%%%%%%%%%%%%%%%%%%%%%%%%%%%%%%%%%%

\newpage


\input{content/Chapters}
% ...more text here.
\section*{Code Python}

\begin{lstlisting}[language=Python]
%matplotlib inline

import numpy as np
import pandas as pd
import statsmodels.api as sm
import statsmodels.formula.api as smf
from statsmodels.tools.sm_exceptions import ConvergenceWarning
import matplotlib.pyplot as plt
from math import sqrt
import seaborn as sns
sns.set_palette("colorblind")

data = pd.read_csv("C:\\Users\Walid\Documents\sleepstudy.csv")
data.index = data[data.columns[0]]
data = data[data.columns[1:4]]

data.head(5)

sns.violinplot(x="Days", y='Reaction', data=data)
plt.savefig("figure.pdf") 

sns.violinplot(x="Days", y='Subject', data=data)
plt.savefig("figure2.pdf") 

# plot the distribution of Reaction
sns.distplot(data.Reaction)
plt.savefig("figure3.pdf")
plt.show()

# plot the distribution of the days
sns.distplot(data.Days, kde=False)
plt.savefig("figure4.pdf") 
plt.show()

sns.lmplot(x = "Days", y = "Reaction", data = data)
plt.savefig("figure5.pdf")

# OLS
modelOLS = smf.ols("Reaction ~ Days", data, groups=data["Subject"])
resultOLS = modelOLS.fit()
print(resultOLS.summary())

# GLM
modelGLM = smf.glm("Reaction ~ Days", data, groups=data["Subject"])
resultGLM = modelGLM.fit()
print(resultGLM.summary())

# LMM
modelLMM = smf.mixedlm("Reaction ~ Days", data, groups=data["Subject"])
resultLMM = modelLMM.fit()
print(resultLMM.summary())

y = data.Reaction
y_predict_LMM = resultLMM.fittedvalues
RMSE_LMM = sqrt(((y-y_predict_LMM)**2).values.mean())
results = pd.DataFrame()
results["Method"] = ["LMM"]
results["RMSE"] = RMSE_LMM

y_predict_GLM = resultGLM.fittedvalues
RMSE_GLM = sqrt(((y-y_predict_GLM)**2).values.mean())
results.loc[1] = ["GLM",RMSE_GLM]

y_predict_OLS = resultOLS.fittedvalues
RMSE_OLS = sqrt(((y-y_predict_OLS)**2).values.mean())
results.loc[2] = ["OLS",RMSE_OLS]

results

performance = pd.DataFrame()
performance["residuals"] = resultLMM.resid.values
performance["Days"] = data.Days
performance["predicted"] = resultLMM.fittedvalues

sns.lmplot(x = "predicted", y = "residuals", data = performance)

ax = sns.residplot(x = "Days", y = "residuals", 
     data = performance, lowess=True)
ax.set(ylabel='Observed - Prediction')
plt.show()


\end{lstlisting}



\end{document}
